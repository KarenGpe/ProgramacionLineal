% Created 2019-04-12 vie 12:05
% Intended LaTeX compiler: pdflatex
\documentclass[11pt]{article}
\usepackage[utf8]{inputenc}
\usepackage[T1]{fontenc}
\usepackage{graphicx}
\usepackage{grffile}
\usepackage{longtable}
\usepackage{wrapfig}
\usepackage{rotating}
\usepackage[normalem]{ulem}
\usepackage{amsmath}
\usepackage{textcomp}
\usepackage{amssymb}
\usepackage{capt-of}
\usepackage{hyperref}
\usepackage{listings}
\author{Karen Guadalupe Lechuga Trejo}
\date{\today}
\title{Programación lineal}
\hypersetup{
 pdfauthor={Karen Guadalupe Lechuga Trejo},
 pdftitle={Programación lineal},
 pdfkeywords={},
 pdfsubject={},
 pdfcreator={Emacs 25.2.2 (Org mode 9.2.3)}, 
 pdflang={English}}
\begin{document}

\maketitle
\tableofcontents


\section{Teoría}
\label{sec:org111e85b}
\subsection{Motivación}
\label{sec:org66a6663}

El objetivo de la programación lineal es es maximizar funciones
lineales sobre dominios convexos, es decir, definidos sobre regiones
dadas por desigualdades.

\begin{center}
\includegraphics[width=.9\linewidth]{Rubens-Saturno-detalle.jpg}
\end{center}

\subsection{Ejemplos}
\label{sec:org011a75a}

\begin{itemize}
\item El problema de la dieta.
\item Optimización de lugares en una excursión.
\item Escoger objetos óptimos para un campamento.
\item El problema del flujo máximo.
\end{itemize}

\subsection{Convexidad}
\label{sec:orgdcf9949}

Un conjunto \(X\) es \textbf{convexo} si para todos \(x,y\in X\) y
\(t\in[0,1]\) se tiene que \(tx + (1-t)y \in X\).
\subsection{Método SIMPLEX}
\label{sec:org3a7334c}

\section{Herramientas computacionales}
\label{sec:org4fb862c}
\subsection{Emacs}
\label{sec:org21464a9}

\begin{center}
\begin{tabular}{ll}
C-x C-s & salvar archivo\\
C-x C-f & abrir archivo\\
M-q & formatear párrafo\\
C-x d & editar directorios\\
C-g & interrumpe procesos\\
C-x 1 & regresa a una sola pantalla\\
C-x 2 & divide horizontalmente\\
C-x 3 & divide verticalmente\\
M-w & copiar la región\\
C-w & borrar la región\\
shift-flechas & seleccionar la región\\
C-y & pegar la región\\
C-c C-e & menú exportar en otros formatos\\
M-flechas & mover renglones/columnas de tabla\\
\end{tabular}
\end{center}

\begin{enumerate}
\item Org mode
\label{sec:orgf905479}
\begin{center}
\begin{tabular}{ll}
C-c C-c & corre un bloque de código\\
C-x b & cambiar buffer\\
\end{tabular}
\end{center}
\end{enumerate}


\subsection{Git}
\label{sec:org5a5e0cf}
\begin{enumerate}
\item Github
\label{sec:org95891dc}
\end{enumerate}
\subsection{Python}
\label{sec:org1e35462}
\begin{enumerate}
\item Lenguaje Python
\label{sec:org48bc0d8}

En el lenguaje Python podemos hacer operaciones:

\lstset{language=Python,label= ,caption= ,captionpos=b,numbers=none}
\begin{lstlisting}
2+2
\end{lstlisting}

También podemos usar la biblioteca pulp.

\lstset{language=Python,label= ,caption= ,captionpos=b,numbers=none}
\begin{lstlisting}
from pulp import *
x = LpVariable("x", 0, 3)
y = LpVariable("y", 0, 1)
prob = LpProblem("myProblem", LpMinimize)
prob += x + y <= 2
prob += -4*x + y
status = prob.solve()
value(x), value(y), value(prob.objective)
\end{lstlisting}

\item Jupyter
\label{sec:org5d60ecd}
\end{enumerate}
\subsection{\LaTeX{}}
\label{sec:org2f96070}
\end{document}